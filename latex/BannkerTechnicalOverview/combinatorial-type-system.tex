\section{Combinatorial analyses of typed programs}

% TODO \mathcal{V} is informal, read more set theory and clean it up.
Let $\mathcal{V}$ be an set called the $variables$, taken as a subset of all sequences from some alphabet $\mathcal{A}$ - the details of this are immaterial for the analysis, but it is helpful to think of $\mathcal{V}$ as a set where, given a finite subset $\mathrm{Bound}\subset\mathcal{V}$, it is easy to find an element $x\in \mathrm{Free} := \mathcal{V} - \mathrm{Bound}$.

In the following we introduce further constructions intended to provide semantic meaning to $\mathcal{V}$. These definitions are inspired by ML-like languages in functional programming, and % TODO:finish this!

\subsection{Inductive type assignments from sorts, axioms, and rules}

Let $\mathcal{S}$ be a set of objects, $\mathcal{S}\cap\mathcal{V}=\emptyset$. 

If $x : s$ and $s\in\mathcal{S}$, we say that $x$ is a \emph{generalized type of sort $S$}. This somewhat weighty term was chosen to avoid confusion around common terms such as ``the type of types.''

Let $\mathcal{A}\subset\mathcal{S}\times\mathcal{S}$ be a set of \emph{axioms} and $\mathcal{R}\subset\mathcal{S}\times\mathcal{S}\times\mathcal{S}$ be a set of \emph{rules}.

\subsubsection{Inductively-defined types from the lowest sort}

Given the triple $\mathcal{S},\mathcal{R},\mathcal{R})$, we can 

\begin{example}
    Let $\mathcal{S} = \{\mathrm{Prop},\mathrm{Type}\}$, $\mathcal{R} = \{(\mathrm{Prop},\mathrm{Type})\}$, and $\mathcal{A} = \{(\mathrm{Prop},\mathrm{Type},\mathrm{Type})\}$.
\end{example}

\subsubsection{Function types and implementations}

\subsubsection{Valid definitons for higher sorts}

\section{Association schemes}

Assocation schemes are central objects in algebraic combinatorics[cite], generalizing notions of regularity in graphs.

\begin{definition}
    Let $X$ be a finite set. Let $\mathrm{Refl} = \{(x,x)\ |\ x\in X\}$. Let $r \subset X\times X$. Define $r^* = \{(y,z)\ |\ (z,y)\in r\}$. For each $x\in X$, define $xr := \{y\in X\ |\ (x,y)\in r \}$.

    Let $A$ be a partition of $X\times X$ with $\emptyset\notin A$ and $\mathrm{Refl}\in A$, and for each $r \in A$, assume $r^* \in A$. Then $(X,A)$ is an \emph{association scheme} if for all $e,f,g\in A$ there exists a positive integer $a_{efg}$ such that for all $y,z\in X$, $(y,z) \in g \Rightarrow | ye \cap zf^{*} | = a_{efg}$.
\end{definition}

\section{Combinatorial type systems}

\begin{definition}
    Let $\mathcal{V}$ be a set, identified as the \emph{variables}, and $\mathcal{S} = \{\mathrm{Rel},\mathrm{Type},\mathrm{Kind}_0,\mathrm{Kind}_1...\}$ be the set of \emph{sorts}. For brevity we will sometimes write $\mathcal{S} = \{\mathrm{Kind}_i, i\geq -1 \}$ where $\mathrm{Kind}_{-1} \equiv \mathrm{Type}$.

    Let $\mathbf{0}$ be the \emph{empty term} and $\mathbf{1}$ be the \emph{unit term}. 

    The set $\mathcal{T}$ of \emph{terms} is defined inductively as folllows:
    \begin{equation*}
        \mathcal{T} = \mathbf{0}\ | \mathbf{1}\ | \mathcal{S}\ |\ \mathcal{V}\ |\ \mathcal{V} : \mathcal{T}\ = \mathcal{T} |\ \mathrm{data}\ \mathcal{V} : \mathcal{T} = \{\mathcal{V} : \mathcal{T}\} \ |\ \mathcal{T} \mathcal{T}
    \end{equation*}

    We define a \emph{typing relation on $\mathcal{T}$ with context $\Gamma\subset \{\mathcal{V}\ :\ \mathcal{T} \}$} as follows. First, say a variable $v\in\mathcal{V}$ is \emph{free in $\Gamma$} if $\forall t\in\mathcal{T},\ v:t\notin \Gamma$. 

    Let $x = \mathrm{data}\ v : t = \{a_i : t_i\}$ and $o = a_i (o_i : t_i)$. The \emph{inductive view of $o$} is defined by
    \begin{equation*}
        \texttt{inductive_view}(0) = \left\{\begin{array}{lr}
            \texttt{base_case}, & $v\ \mathrm{free\ in\ } t_i$\\
            \textt{ind}(o':t), & \mathrm{otherwise} 
            % for cases like data MyType of Conses int*MyType*MyType, we will have to fix a in-out,l-r order
            % E.g. Conses (1,Conses 2 BaseCase BaseCase,Conses 3 BaseCase (Conses 4 BaseCase BaseCase)) ->
            % Conses (1, BaseCase, Conses 3 BaseCase (Conses 4 BaseCase BaseCase))
            % Conses (1, BaseCase, (Conses 4 BaseCase BaseCase))
            % Conses (1, BaseCase, BaseCase)
            % BaseCase
            % Need to make sure map is well-defined. Not a trivial question.
            % Actually - what we want is to look at the basis for the free group of the data V : T {V : T} terms.
            % It *seems* like these terms should always be finitely generated for any context and (finite) data definition.
            % data Nat =
            % | Z of 1
            % | S of Nat
            % data FreeGroupofType : Type -> 
        \end{array}
            \right }
    \end{equation*}

    
    \begin{itemize}
        \item \textbf{Empty} 
        \begin{equation*}
        \frac{ }{\Gamma \vdash \mathbf{0}:\mathrm{Type},\ \mathbf{1}:\mathrm{Type},\ \mathbf{1} : \mathbf{1},\ \mathrm{Rel} : \mathrm{Type},\  \mathrm{Type} : \mathrm{Kind}_0,\ \mathrm{Kind}_i : \mathrm{Kind}_{i+1}}
        \end{equation*}
        \item \textbf{Intro}
        \begin{equation*}
            \frac{v,w,x,y...\mathrm{\ free\ in}\ \Gamma,\ \Gamma \vdash A,B,C..., \Gamma \vdash s\in \mathcal{S}}
            {\Gamma \vdash \left(\mathrm{data}\ v\ :\ s\ = \{w\ :\ A,\ x\ :\ B  \}\right) }
        \end{equation*} 
        \item \textbf{Instance}
        \begin{equation*}
            \frac{v\mathrm{\ free\ in}\ \Gamma,\Gamma\vdash(\mathrm{data}\ v\ :\ s\ = X), \{w\ :\ A\}\in X,\Gamma\vdash a:A}
            {\Gamma\vdash v = w a}    
        \end{equation*}
        \item \textbf{Product}
        \begin{equation*}
            \frac{\Gamma \vdash A:\mathrm{Kind}_i, x:A, B:\mathrm{Kind}_{j},v \ \mathrm{free\ in\ }\Gamma,j\in\{i,i+1\}}
            {\Gamma \vdash (\mathrm{data}\ v\ (A ) )}
        \end{equation*}
        \item \textbf{Pattern}
    \end{itemize}
\end{definition}

Let $(\mathcal{T},\Gamma)$ be a combinatorial type system. 

Define $\mathrm{inductiveView}_{\Gamma} : \mathcal{T}\rightarrow\mathcal{T}$ by
%         \mathcal{T} = \mathbf{0}\ | \mathbf{1}\ | \mathcal{S}\ |\ \mathcal{V}\ |\ \mathcal{V} : \mathcal{T}\ = \mathcal{T} |\ \mathrm{data}\ \mathcal{V} : \mathcal{T} = \{\mathcal{V} : \mathcal{T}\} \ |\ \mathcal{T} \mathcal{T}
% $\mathcal{S} = \{\mathrm{Rel},\mathrm{Type},\mathrm{Kind}_0,\mathrm{Kind}_1...\}$
\begin{equation*}
    \mathrm{inductiveView}_{\Gamma} \mathbf{0} = \mathbf{0}
    \mathrm{inductiveView}_{\Gamma} \mathrm{1} = \mathbf{1}
    \mathrm{inductiveView}_{\Gamma} \mathrm{Type} = \mathbf{1}
    \mathrm{inductiveView}_{\Gamma} \mathrm{Kind}_0 = \mathrm{Type}
    \mathrm{inductiveView}_{\Gamma} \mathrm{Kind}_{i+1} = \mathrm{Kind}_i \ \forall i\geq 0
    \mathrm{inductiveView}_{\Gamma} v = \mathbf{1}\ \forall v\in\mathcal{V}
    \mathrm{inductiveView}_{\Gamma} v:t_1=t_2 = 
\end{equation*}

\subsection{Simple constructions for the empty context}

\section{Overview of pure type systems}

We present our definition of a \emph{computational} pure type system with \emph{definitions}, loosely following the manner of extensions in \cite{ClassicalPTS97} and \cite{PTSWithDefinitions}, from the classic \cite{BarendregtPTS1}. We also integrate ideas from \cite{RhoCalculus}.

 
\begin{definition}\label{puretypesysdef}
    A \emph{pure type system} is a tuple $(V,S,\mathcal{U},\mathcal{D},\Gamma,\cong)$
    \begin{itemize}
        \item $V$ is a set of objects called \emph{variables},
        
%        \item $C$ is a set of objects called \emph{constants}, with $V \cap C = \emptyset$, 
% We probably won't need to treat constants as anythng other than variables the compiler recognizes.
% If we change our mind, be sure to put it back in \mathcal{U}, etc        
        \item $S$ is a triple $(\mathcal{S},\mathcal{A},\mathcal{R})$ where
        \begin{itemize}
            \item $\mathcal{S}$ is a set of objects called \emph{sorts}.
            \item $\mathcal{A} \subset \mathcal{S}\times\mathcal{S}$, and is called the \emph{axioms}.
            \item $\mathcal{R} \subset \mathcal{S}^3$, and is called the \emph{rules}.
        \end{itemize}
        
        \item $\mathcal{U}$ is a set of \emph{pseudoterms}, defined inductively by 
            \begin{equation*}
                \mathcal{U} = V\ |\ \mathcal{S}\ |\ \mathcal{U}\ \mathcal{U}\ |\ \lambda V : \mathcal{U}\ .\ \mathcal{U}\  %|\ \Pi V : \mathcal{U} .\mathcal{U}
            \end{equation*}
        If $x \in \mathcal{V}$, we write $b(x) \in \mathcal{U}$ to mean that $b = x$ as elements of $\mathcal{V}$, or that $x$ appears in either an applicative ($\mathcal{U} \mathcal{U}$) or $\lambda$ construction.

        \item $\Gamma \subset \{x : A\ |\ v\in\mathcal{V},A \in \mathcal{U}\}$ is a \emph{context}, defined inductively by the following rules:
            \begin{itemize}
                \item \textbf{Empty} : If $(r_1,r_2)\in\mathcal{R}$, then $r_1 : r_2 \in \Gamma$
                \item \textbf{Start} : Given $s\in\mathcal{S}$ and $x : s \in \Gamma$, then for all $v \in \mathcal{V} - \{a\ |\ a \in \mathcal{V} \land \exists A\in\mathcal{U} \ \mathrm{such \ that\ } a : A \in \Gamma \}$, $v : x \in \Gamma$.
                \begin{itemize}
                    \item The set $\mathcal{V} - \{a\ |\ a \in \mathcal{V} \land \exists A\in\mathcal{U} \ \mathrm{such \ that\ } a : A \in \Gamma \}$ is called the \emph{fresh variables of} $\Gamma$. 
                \end{itemize}
                \item \textbf{Equality}
                \item \textbf{Rewrite}
                    %\begin{equation*}
                    %    
                    %\end{equation*}
            \end{itemize}
        
            \item $\cong$ is an equivalence relation on psuedoterms, encompassing the reflexive, transitive, and symmetric closures of the following well-known relations:
            \begin{itemize}
                \item \textbf{Syntactic equality}: Objects in $V$, $C$, and $\mathcal{S}$ retain their ``built-in'' notion of equality (e.g. set-theoretic), which is extended component-wise into inductively-defined elements of $\mathcal{U}$. We denote this as $a \equiv b$.
                \begin{itemize}
                    \item It is important to note that syntactic equality exists \emph{metatheoretically}. As a trivial but illustrative example, suppose we have $\texttt{Type} \in \mathcal{S}$ and $\texttt{Nat},\texttt{nat},+ \in \mathcal{V}$. Let $a = \lambda x:\texttt{Nat} . (x + x)$ and $b = \lambda x: \texttt{nat} . (x + x)$. If our programming language's parser is case-sensitive, then these are different terms and $a !\equiv b$ (e.g. $\texttt{Nat}$ and $\texttt{nat}$ are different types). However, for purely user-facing reasons, we might have a case-insensitive parser, in which case these are literally the same object and $a \equiv b$.
                    \item The relation $\equiv$ is an \emph{assertion}, not a \emph{judgment}. Its validity cannot be checked in any pure type system, but rather by inspecting the source code of the parser, checking  asking the editor if a seeming mistake is a typographical error, and so on.
                    \item While this seems pedantic, clarity about the notion of equality is % TODO: FILL THIS IN!!!
                \end{itemize} 
                \item \textbf{$\alpha$-reduction}: Given an element $x = \lambda v : a . b(x)$ of $\lambda V : \mathcal{U}\ .\ \mathcal{U}$, then $x \rightarrow_{\alpha} y$ if $y = \lambda u : a . b(u)$. That is, $\alpha$-reduction expresses that function objects are ``the same function'' if all we are doing is changing the name of the variable.
                %$\alpha$ reduction is defined similarly for  $\Pi V : \mathcal{U} .\mathcal{U}$ terms.
                \item \textbf{$\beta$-reduction}
                \item \textbf{$\eta$-reduction}
                \item \textbf{Equality types}: Given terms $x : A, y : A$, the type $(x \cong_A y)$, written $(x = y)$, is defined in $\Gamma$, generated by the rule $x \cong y \rightarrow \mathrm{Refl} x x : (x = y)$.
            \end{itemize}
    \end{itemize}
\end{definition}

\section{Equality in pure type systems}



\section{The type-adjacency graph of a PTS}

\section{The free group of definitions from a pure type system}

Let $\mathcal{U}$ be the set of pseudoterms for a pure type system with sorts $\mathcal{S}$. Let $\mathrm{Free}\lbrack V , \mathcal{U}\rbrack$ be the set of all finite products $(V \times \mathcal{U})^n$.
Am element $x . y u : s$ of $\mathcal{D}$, called a \emph{definition}, has a \emph{sort} $s$, \emph{type} $x$, and constructor $y u$. 

% Example to illustrate what I am getting at
% let MyType : Type =
%   | Standard
%   | IntIndexed of Int
%   | StrIndexed of String
% In \mathcal{D} an element here might be (MyType . Standard \emptyset : Type) or (MyType . IntIndexed 5 : Type)
% In a "free group" sense we have "sort inverses" Type^{-1}, "type inverses" MyType^{-1}, "constructor inverses" Standard^{-1} and IntIndexed^{-1}, and "argument inverses" 5^-1.
% Some programming languages seem natural out of this: for instance, pattern matching is like applying type/sort inverses. 
% With respect to the context, you know you are "in" the sort of Types and "in" the type 
% Can the typechecking rules be integrated into this group structure? How does ths related to strong normalization?
% Perhaps we can view typechecking as group actions on the free group generated by the actual definitions in the context?
% Substituton => inverse action of (\lambda x: f(x))[x=a] is just dropping the substitution
% Is substitution associative? What even is the action?
% A group element a \in FG[V,U]...
% Trying to reconcile the lambda / pi calculus - this seems like the rho calculus
% only enriched with algebraic structure inhereted from PTS?
% The idea of the rho calculus is an extension of syntactic identity and pattern matching
% to a depdent type theory without separate pi / lambda constructors.


\section{Combinatorial homotopies}

We largely follow \cite{CombinatorialHomotopyLectures} % https://www-fourier.ujf-grenoble.fr/~sergerar/Papers/Trieste-Lecture-Notes.pdf

\section{The combinatorial type system}

\section{The typing graph $\mathbb{T}(\mathcal{T})$}

\begin{definition}
    Let $\mathcal{T},\Gamma$ be a set of terms and a typing context, as above. The \emph{typing graph} $\mathbb{\Pi}(\mathcal{T})$ is the directed graph with vertex set $\mathcal{T}$ and edges $E$ defined by $E = \{(x,y)\ | \ \Gamma \vdash x:y \}$.
\end{definition}

\section{The dual typing graph $\mathbb{U}(\mathcal{T})$}

\begin{definition}
    Let $\mathcal{T},\Gamma$ be a set of terms and a typing context, as above, with sorts $\mathcal{S}$. The \emph{dual typing graph} $\mathbb{T}(\mathcal{T})$ is the directed graph with vertex set $\mathcal{T}$ and edges $E$ defined by $E = \{(x,y)\ | \ \Gamma \vdash x:a:s, y:(a \rightarrow s) \ \}$.
\end{definition}

\section{Commentary}

One somewhat amusing subjective aspect of using naive finite set theory and combinatorics is that, in the author's experience, it is quite easy to make ``obvious'' errors that are nevertheless ``well-typed'' in the metatheory, which can lead to antecedent errors in reasoning 

On the

This experience is familiar to software developers